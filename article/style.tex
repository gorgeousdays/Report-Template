% 支持中文的 ctex 宏包
\RequirePackage{ctex}
% 页面布局
\RequirePackage{geometry}
% 数学宏包
\RequirePackage{amsmath}
\RequirePackage{amsfonts}
\RequirePackage{amssymb}
\RequirePackage{bm}
% 设置字体
\RequirePackage{fontspec}
% 设置颜色
\RequirePackage{xcolor}
% 插入图片
\RequirePackage{graphicx}
% 表格
\RequirePackage{array}
%% 长表格
\RequirePackage{longtable}
% 首行缩进
\RequirePackage{indentfirst}
% 图片浮动
\RequirePackage{float}



% 每行缩进两个汉字
\setlength\parindent{2em}
% 解决中英文混排文字超出边界问题
\sloppy
%设置页码
\pagenumbering{arabic}



%用于绘制下划线
\makeatletter
\newcommand\dlmu[2][4cm]{\hskip1pt\underline{\hb@xt@ #1{\hss#2\hss}}\hskip3pt}
\makeatother



%使\section中的内容左对齐
\makeatletter
\renewcommand\thesection{\chinese{section}、\kern-1em}
\renewcommand\thesubsection{\arabic{section}\thinspace.\thinspace\arabic{subsection}}
\renewcommand\thesubsubsection{\thesubsection\thinspace.\thinspace\arabic{subsubsection}}
\renewcommand\section{\@startsection{section}{1}{\z@}%
 	{-3.5ex \@plus -1ex \@minus -.2ex}%
	{2.3ex \@plus.2ex}%
	{\bf\large\leftline}}
\renewcommand\subsection{\@startsection{subsection}{2}{\z@}%
	{-3.25ex\@plus -1ex \@minus -.2ex}%
	{1.5ex \@plus .2ex}%
	{\normalfont\large\bfseries\leftline}}
\renewcommand\subsubsection{\@startsection{subsubsection}{3}{\z@}%
	{-3.25ex\@plus -1ex \@minus -.2ex}%
	{1.5ex \@plus .2ex}%
	{\normalfont\normalsize\bfseries\leftline}}
\renewcommand\paragraph{\@startsection{paragraph}{4}{\z@}%
	{3.25ex \@plus1ex \@minus.2ex}%
	{-1em}%
	{\normalfont\normalsize\bfseries\leftline}}
\renewcommand\subparagraph{\@startsection{subparagraph}{5}{\parindent}%
	 {3.25ex \@plus1ex \@minus .2ex}%
	 {-1em}%
	 {\normalfont\normalsize\bfseries}}
\makeatother



%页面设置
\setlength{\topmargin}{5pt}
\setlength{\headheight}{5pt}
\setlength{\headsep}{5pt}
\setlength{\footskip}{30pt}
\setlength{\voffset}{-5pt}
\setlength{\hoffset}{16pt}
\setlength{\oddsidemargin}{0pt}
\setlength{\evensidemargin}{\oddsidemargin}
\setlength{\marginparpush}{0pt}
\setlength{\marginparwidth}{0pt}
\addtolength{\textheight}{4\baselineskip}



%代码格式
\usepackage[toc,page]{appendix}
\usepackage{listings}
\definecolor{dkgreen}{rgb}{0,0.6,0}
\definecolor{gray}{rgb}{0.5,0.5,0.5}
\definecolor{mauve}{rgb}{0.58,0,0.82}
\lstset{ %
	%	language=Python,                % the language of the code
	breaklines,%自动折行
	%extendedchars=false%解决代码跨页时,章节标题,页眉等汉字不显示的问题
	keepspaces=false,  
	%tabsize=4 %设置tab空格数
	showspaces=false,  %不显示空格
	showtabs=false,  
	showstringspaces=true, 
	numbers=left, 
	basicstyle=\footnotesize, 
	numberstyle=\tiny, 
	numbersep=5pt, 
	keywordstyle= \color{ blue!70},%关键字颜色
	commentstyle= \color{red!50!green!50!blue!50},%注释颜色 
	frame=shadowbox, % 边框格式:阴影效果
	rulesepcolor= \color{ red!20!green!20!blue!20} ,
	escapeinside=``, % 英文分号中可写入中文
	xleftmargin=2em,xrightmargin=2em, aboveskip=1em,%设置页边距
	framexleftmargin=2em
}





